\begin{figure}[!htb]
\centering
\begin{tikzpicture}[scale=2]
\draw[color=black, thick]

%Linea de Tierra
(0,0) to (1,0){} 

% NODO A TIERRA
(0.5,0) node[ground]{}

% Paralelo de R1 y C1
(0,0) to [R,l=$R$] (0,1.5)
(1,0) to [C, l=$C$] (1, 1.5)--(0,1.5)

%Serie de R1 y C1
(1,1.5) to [R, l_=$R$, *-] (2, 1.5) to [C, l_=$C$] (3, 1.5)

%Amplificador
(2,3) node[ op amp ] (opamp) {}
(opamp.out) node[above] {\Large $v_{osc}$}
(opamp.+) node[above] {\Large $v_f$}
(opamp.+)-| (1,1.5)
(opamp.-)-| (1, 4.25) to[R,l=$R_2$,*-*] (3, 4.25) -- (3,3)
(opamp.-)-|(0,3) to[R, l=$R_1$] (0,2.3)
% NODO A TIERRA
(0,2.3) node[ground]{}
(1,4.25)-| (1, 5) 
[zD*,l_=$D_1$] (2, 5) to (1,5)
(2,5) to [zD*,l^=$D_2$] (3,5)--(3,3)
(opamp.out) -| (3,1.5)
(opamp.up) --++ (0,0.25) node[above]{15\,\textnormal{V}}
(opamp.down) --++(0,-0.25) node[below]{-15\,\textnormal{V}}

;
\draw[black,thick,dashed] (-0.45,1.9) rectangle (3.25,-0.35);

\draw[black,thick,dashed] (-0.45,4.8) rectangle (3.25,2);
\end{tikzpicture}
\caption{Oscilador de Wien}
\label{fig:oscilador}
\end{figure}